% Template created by Adrienne Traxler (adrienne.traxler@wright.edu)
% Last modified: 3/22/17: Add clarifying note about why the template has too much space around its section headers.
% 
% Changelog
% 3/14/17: add 2-column figure example, tweak text to align with Word sample file, add hyperref for links/cross-references
% 6/21/16: 	Fix old PACS option and superscript affiliations.
% 5/5/16: 	Update sample table to use the REVTEX ruledtabular environment for auto-sizing;
%   		switch to using pra option until the prstper style is fixed (who knows when).
% 5/10/16: Swap suggested department/institution ordering in \affiliation lines.
% 5/12/16: Remove PACS line (deprecated).
% 6/21/16:	Switch to noshowpacs and add superscriptaddress in documentclass options.
% 
% This is my best effort to follow the Physical Review style guide plus specific changes 
% required for PERC (mostly, omitting article titles in references). This version compiles 
% with pdfLaTeX alone; using a proper .bib file changes the bibliography part at the end 
% and would require running BibTex as well.
%
% Finally, there's extra padding around section headers if you compile the bare template. 
% I believe that's because LaTeX stretches its white space to keep the floats (tables and 
% figures) near their input locations. The excess spacing goes away when a normal amount 
% of body text is filled in.

% Big list of reference documentation is here: http://journals.aps.org/revtex, 
%  see especially the APS Author Guide PDF link.

% Notes on the revtex4-1 documentclass options used: 
%  reprint does the double-column, publication ready appearance
%  prstper style formats reference numbers incorrectly, fast fix is to use pra instead.
%  Add longbibliography option to show article titles (and see note in bibliography)

%\documentclass[english,aps,prstper,reprint,showpacs]{revtex4-1}   % This version should work (prstper option), but actually formats reference numbers incorrectly.
\documentclass[english,aps,pra,reprint,noshowpacs,superscriptaddress]{revtex4-1}   
% Using pra instead of prstper gives correct square-bracket (not superscript) reference formatting
\usepackage[T1]{fontenc}	% should generally be included for better accented-word behavior
\usepackage[latin9]{inputenc}	% should generally be included for better accent behavior
\usepackage{geometry}		% for controlling page margins
\geometry{verbose,tmargin=1in,bmargin=1in,lmargin=0.75in,rmargin=0.75in}	% define margins
\usepackage{graphicx}
\usepackage[above,below]{placeins}	% allows use of \FloatBar­rier command to force section barriers
\usepackage{times}
% Next three lines are optional, use the hyperref package to make URLs and reference links live.
\usepackage{hyperref}  
\hypersetup{colorlinks=true,urlcolor=blue,citecolor=blue,linkcolor=blue}   
\urlstyle{same}
\pagestyle{empty}			% page numbers added later, when compiling the whole proceedings
\begin{document}

\title{Paradigms in Physics 2.0}
\author{David Roundy}
\author{Liz Gire}
\author{Ethan Minot}
\author{Emily van Zee}
\affiliation{Department of Physics, Oregon State University, Corvallis, Oregon, 97331}
\author{Tevian Dray}
\affiliation{Department of Mathematics, Oregon State University, Corvallis, Oregon, 97331}
\author{Corinne A. Manogue}
\affiliation{Department of Physics, Oregon State University, Corvallis, Oregon, 97331}

%\keywords{}

\begin{abstract}
In 2016, the Department of Physics at Oregon State University began a
process to revise our Paradigms in Physics curriculum for physics
majors.  This poster will describe both the process by which our
department achieved consensus on this curricular change, and the
resulting curriculum.  The Paradigms 2.0 committee begain with a
survey of students and faculty, followed by individual interviews with
the faculty teaching each existing course.  As we developed a plan to
address student- and faculty-identified challenges in the curriculum,
we met with each faculty member individually to explain and refine our
proposal, which was unanimously approved by the faculty.  Major
changes include elimination or major changes to several courses (math
methods, computational physics, modern physics, electronics, and
classical mechanics), including the introduction of two sophomore-year
courses designed specifically to help prepare students to for their
upper-division courses.
\end{abstract}

\maketitle

\section{Introduction}
The Paradigms in Physics program

\subsection{Background}
\emph{What is the Paradigms now?}
\subsection{History of the Paradigms}
\emph{How did we get here?~\cite{manogue2001paradigms}}
\subsection{Motivation for the change}
\emph{Why change? Why now? Should you do the same?}

\section{Paradigms 2.0 process}
\emph{How did we do it?}

\section{Changes made}
\emph{What did we change?}

\subsection{Math bits}
\emph{What to do with math methods?}
\subsection{Electronics}
\emph{Reduce quantity of electronics required, clarify expectations.}
\subsection{5-week Paradigms}
\emph{Why lengthen it?}
\subsection{Sophomore courses}
\emph{Give students more preparation when possible.  Replace Modern
  Physics and Classical Capstone courses.}
\subsection{Reordering}
\emph{Improve interaction with requirements.}

\section{Outcomes}
\emph{How has it gone?}

\subsection{Faculty vote}
\emph{Unanimous result.}
\subsection{Transition year}
\emph{Gradual transition, getting courses taught.}

\section{Future research}
\emph{Document learning trajectories, new courses.}

\section{Tables and Figures}

Figures, tables, and equations must be inserted in the text and may
not be grouped at the end of the paper. Important: miscounting of
figures, tables, or equations may result from revisions. Please double
check the numbering of these elements before you submit your paper to
the volume editor.
% LaTeX handles this automatically via the \label and \ref commands.

\subsection{Figures}

If your manuscript contains figures, they are typically placed in one
column (see Fig.\ \ref{fig1}).  A large figure might span two columns
(see Fig.\ \ref{fig2}).  Cite all figures in the text
consecutively. The word ``Figure'' should be spelled out if it is the
first word of the sentence and abbreviated as ``Fig.'' elsewhere in
the text. Place the figures as close as possible to their first
mention in the text at the top or bottom of the page.

\begin{figure}
%\includegraphics[width=0.8\linewidth]{figure1}
\caption{Figure captions appear below the figure while table captions appear above the table.\label{fig1}}
\end{figure}


\subsubsection{Color figures}
Beginning in 2013, color figures are allowable since the PERC
Proceedings will be available only in an online format.  However,
authors are encouraged to check that the figures print in black and
white without loss of clarity.

% Note: Revtex puts two-column figures at the top of a page. If you really need 
% to change this, gird yourself for battle and see http://tinyurl.com/j2wnoxb.
\begin{figure*}
  %\includegraphics{figure2}
  picture here of the new curriculum
\caption{A 2-column figure. Center figure captions if they run one
  line only, and justify captions if they are multi-line.\label{fig2}}
\end{figure*}

% REVTeX provides a ruledtabular environment that automatically sizes tables, adds double 
% lines, and adjusts column spacing. Use it around the regular tabular environment.
% For a two-column table, use \begin{table*} instead (with matching \end{table*}.
\begin{table}[htbp]
\caption{This is a sample table.\label{tab1}}
\begin{ruledtabular}
\begin{tabular}{cl}
 & \textbf{MS Word} \\
 & \textbf{US Letter Size Paper} \\
 \hline
Primary Heading & Roman numerals, all capital letters \\
First subheading & Capital letters, only first \\
 & letter of first word capitalized \\
Second subheading & Numbers, italicized 
\end{tabular}
\end{ruledtabular}
\end{table}

% Formatting tweak if needed--FloatBarrier forces floats to show here, before next section
%\FloatBarrier	

\section{Other issues}

\subsection{Copyright transfer}
All accepted papers will be made available under a Creative Commons Attribution 3.0 License (CC-BY).

\subsection{Citations}
When you are preparing references, please use the PRST--PER Style
Guide~\cite{style-guide}.  This document contains information about
how references should be formatted, including proper journal
abbreviations.  Examples directly from that style guide of a journal,
book, proceedings paper, and dissertation are also provided
here~\cite{smith-brown,smith,smith-proc,smith-diss}.

\subsection{Permissions}
To use previously published material from a book or journal, you must
obtain written permission from the owner of the rights to the material
(the original publisher and/or author). It is your responsibility to
obtain permission to use copyrighted material. The executed
permissions need to be sent along with the manuscript to your volume
editor. Most publishers offer submission of permission requests online
or via email, which may be the fastest and most convenient way of
receiving a reply. Some examples with relevant links are:

\begin{itemize}
\item \url{http://www.elsevier.com/locate/permissions}
\item \url{http://www.ieee.org}
\item \url{http://www.nature.com}
\item \url{http://www.sciencemag.org}
\end{itemize}

You may also use the Permission Request Form to request permission to
reprint text, tables or figures. You may complete this form and fax it
to the publisher or author of the material you wish to use. A blank
form is available for download click on Forms. When the signed
permission is returned to you, please insert any necessary credit
lines in your figure or table legends.


\section{Conclusions}
This template was newly updated for the 2016 PERC Proceedings.  The
editors apologize if any errors exist, and encourage you to contact
them with changes and other suggestions.  

\acknowledgments{Put references below the acknowledgements (and appendixes, if any).}

% For a longer bibliography, delete the thebibliography block above, then comment in 
% these two lines to use a .bib file with BibTeX.
\bibliographystyle{apsrev}  	% supercedes the longbibliography option, so leave commented out if you want to display article titles
\bibliography{paper}  	% don't include the .bib suffix

\end{document}
